\titledquestion{Disjoint Set practice}

Let's performing a series of merge opertions on a disjoint set structure.
Please show the final disjoint set tree for each of the following optimization strategies.

\newcommand{\SET}[1]{\{#1\}}

\begin{enumerate}[{\bf op} 1.]
        \item initialize: $\SET{1}, \SET{2}, \SET{3}, \SET{4}, \SET{5}, \SET{6}, \SET{7}, \SET{8}, \SET{9}$
        \item merge 1,4
        \item merge 3,2
        \item merge 7,6
        \item merge 7,3
        \item find 2
        \item merge 5,8
        \item merge 1,9
        \item merge 9,8
        \item merge 9,2
        \item find 6
\end{enumerate}

\begin{parts}
        \part[2] Only with union-by-height optimization. (When two trees have the same height, the set specified first in the union will be the root of the merged set.)

        \begin{solution}
                \begin{center}
                        \begin{tikzpicture}[level distance=1.5cm,
                                        level 1/.style={sibling distance=3cm},
                                        level 2/.style={sibling distance=1.5cm}]
                                \node {1}
                                child {node {5}
                                                child {node {8}}}
                                child {node {7}
                                                child {node {3}
                                                                child {node {2}}}
                                                child {node {6}}}
                                child {node {9}}
                                child {node {4}};
                        \end{tikzpicture}
                \end{center}
        \end{solution}


        \part[2] Only with path compression (The set specified first in the union will always be the root of the merged set).

        \begin{solution}
                \begin{center}
                        \begin{tikzpicture}[level distance=1.5cm,
                                        level 1/.style={sibling distance=3cm},
                                        level 2/.style={sibling distance=1.5cm}]
                                \node {1}
                                child {node {5}
                                                child {node {8}}}
                                child {node {7}
                                                child {node {3}}
                                                child {node {2}}}
                                child {node {9}}
                                child {node {4}}
                                child {node {6}};
                        \end{tikzpicture}
                \end{center}
        \end{solution}
\end{parts}

