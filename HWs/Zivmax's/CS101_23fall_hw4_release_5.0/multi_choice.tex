\titledquestion{Multiple Choices}

Each question has \textbf{one or more} correct answer(s). Select all the correct answer(s). For each question, you will get 0 points if you select one or more wrong answers, but you will get 1 point if you select a non-empty subset of the correct answers.

Write your answers in the following table.

\begin{table}[htbp]
    \centering
    \begin{tabular}{|p{2cm}|p{2cm}|p{2cm}|p{2cm}|}
        \hline
        (a) & (b) & (c) \\
        \hline
        %%%%%%%%%%%%%%%%%%%%%%%%%%%%%%%%%%%%%%%%%%%%%%%%%%%%%%%%%%
        % YOUR ANSWER HERE.
            &     &     \\
        %%%%%%%%%%%%%%%%%%%%%%%%%%%%%%%%%%%%%%%%%%%%%%%%%%%%%%%%%%
        \hline
    \end{tabular}
\end{table}

\begin{parts}
        
    \part[3] Which of the following sorting algorithms can be implemented as the stable ones?
    \begin{choices}
        \CorrectChoice  Insertion-Sort 
        \CorrectChoice Merge-Sort
        \choice Quick-Sort (always picking the first element as pivot)
        \choice None of the above
    \end{choices}
    
    \part[3] Which of the following implementations of quick-sort take \(\Theta(n\log n)\) time in the \textbf{worst case}?
    
    \begin{choices}
        \choice Randomized quick-sort, i.e. choose an element from \(\left\{a_l,\cdots,a_r\right\}\) randomly as the pivot when partitioning the subarray \(\langle a_l,\cdots,a_r\rangle\).
        \choice When partitioning the subarray \(\langle a_l,\cdots,a_r\rangle\) (assuming \(r-l\geqslant 2\)), choose the median of \(\left\{a_x,a_y,a_z\right\}\) as the pivot, where \(x,y,z\) are three different indices chosen randomly from \(\{l,l+1,\cdots,r\}\).
        \choice When partitioning the subarray \(\langle a_l,\cdots,a_r\rangle\) (assuming \(r-l\geqslant 2\)), we first calculate \(q = \frac{1}{2} (a_{\max} + a_{\min})\) where \(a_{\max}\) and \(a_{\min}\) are the maximum and minimum values in the current subarray respectively. Then we traverse the whole subarray to find \(a_m \; s.t. \left|a_m - q\right| = \min_{i=l}^r \left|a_i - q\right|\) and choose \(a_m\) as the pivot.
        \CorrectChoice None of the above.
    \end{choices}

    \part[3] Which of the following statements are true?
    
    \begin{choices}
        \CorrectChoice If \(T(n) = 2T(\frac{n}{2}) + O(\sqrt{n})\) with \(T(0) = 0\) and \(T(1) = 1\), then \(T(n) = \Theta(n)\).
        \choice If \(T(n) = 4T(\frac{n}{2}) + O(n^2)\) with \(T(0) = 0\) and \(T(1) = 1\), then \(T(n) = \Theta(n^2 \log n)\).
        \CorrectChoice If \(T(n) = 3T(\frac{n}{2}) + \Theta(n^2)\) with \(T(0) = 0\) and \(T(1) = 1\), then \(T(n) = \Theta(n^2)\).
        \CorrectChoice If the run-time $T(n)$ of a divide-and-conquer algorithm satisfies \(T(n) = aT(\frac{n}{b}) + f(n)\) with \(T(0) = 0\) and \(T(1) = 1\), we may deduce that the run-time for merging solutions of $a$ subproblems of size $\frac{n}{b}$ into the overall one is $f(n)$.      
    \end{choices}

\end{parts}CorrectChoice