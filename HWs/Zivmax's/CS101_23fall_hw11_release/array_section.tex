\titledquestion{Array Section: Maximal Power}

Given a sequence of positive integers $A=\langle a_1,\cdots,a_n\rangle$, we want to divide it into several consecutive sections so that the sum of power of these sections
is maximized.

The power of section $\langle a_i,\cdots,a_j\rangle$ is defined as $aX_{i,j}^2+bX_{i,j}+c$, where
\begin{itemize}
    \item $a,b,c$ are given constants.
    \item $X_{i,j}=\sum\limits_{k=i}^j a_k$ is the sum of values in the section.
\end{itemize}

Please design a \textbf{dynamic programming} algorithm that returns the maximal sum of power of the sections that you divided.

For example, if \(a=-1,b=10,c=-20\) and \(A=\langle 2,2,3,4\rangle\), the maximal sum of power is \(9\), and the way to divide the sequence is: \(\langle 2,2\rangle,\langle 3\rangle,\langle 4\rangle\).

\begin{parts}
    \part[3] Define the subproblems for $i\in[0,n]$: $OPT(i)=$ the maximal sum of power of if you only consider dividing the first $i$ elements. Give your Bellman equation to solve the subproblems.
    \begin{solution}
        $$
            OPT(i) = \begin{cases}
                0                                                                 & \text{if } i = 0 \\
                \max_{1 \leq j \leq i} \{ OPT(j-1) + aX_{j,i}^2 + bX_{j,i} + c \} & \text{if } i > 0 \\
            \end{cases}
        $$
    \end{solution}

    \part[1] What is the answer to this question in terms of $OPT$?
    \begin{solution}
        The answer to the problem is $OPT(n)$.
    \end{solution}

    \part[3] What is the runtime complexity of your algorithm? (answer in $\Theta(\cdot)$)
    
    \textbf{Hint:} How will you compute $X_{j,i}$ in $\Theta(1)$ time? Please give your analysis of the preprocessing and computing complexity.
    \begin{solution}
        To calculate the runtime complexity, we consider the number of subproblems and the time it takes to solve each subproblem. We have $n$ subproblems corresponding to $OPT(1), OPT(2), \ldots, OPT(n)$. For each subproblem $OPT(i)$, we must compute the maximum over $i$ possible sections. The complexity of computing each section's power is $O(1)$ after preprocessing, so the complexity for each $OPT(i)$ is $O(i)$.
        
        The total complexity is the sum of the complexities for all subproblems:
        
        $$
        \sum_{i=1}^{n} O(i) = O\left(\frac{n(n+1)}{2}\right) = O(n^2)
        $$
        
        Thus, the runtime complexity of the algorithm is $\Theta(n^2)$.
                
        \pagebreak
        
        \textbf{Preprocessing and Computing Complexity:}

        To compute $X_{j,i}$ in $\Theta(1)$ time, we can preprocess the input sequence to create a prefix sum array $P$, where $P[i] = \sum_{k=1}^{i} a_k$. The preprocessing step takes $\Theta(n)$ time.

        Once we have the prefix sum array, we can compute $X_{j,i}$ using $P[i] - P[j-1]$ in constant time.

        The preprocessing complexity is $\Theta(n)$, and the overall complexity remains $\Theta(n^2)$ since this does not dominate the $\Theta(n^2)$ complexity of solving the subproblems.
    \end{solution}
\end{parts}



